\documentclass{article}
\usepackage[utf8]{inputenc}
\usepackage{tikz}
\usepackage{amsmath}

\title{Week 16}
\date{\today}

\begin{document}

\maketitle

\section{Rooted trees}

\title{Outlines}

\begin{itemize}
  \item Definition of rooted trees
  \item Terminology of rooted trees
  \item Depth and height in a tree
  \item Special trees
  \item Regular rooted trees and properties
  \item Isomorphic rooted trees and properties
\end{itemize}

\subsection{Definition of rooted trees}

\paragraph{A rooted tree is a directed tree having one distinguished vertex r, called a root, such that for every vertex r there is a directed path from r to v.}

\begin{center}
  \begin{tikzpicture}
      \node {r}
        child {node {a}}
        child {node {b}
          child {node {c}}
          child {node {d}}
        }
        child {node {e}
          child {node {f}}
          child {node {j}}
        };
    \end{tikzpicture}
\end{center}
  

\subsection{Terminology of rooted trees}
\subsection{Depth and height in a tree}
\subsection{Special Trees}
\subsection{Regular rooted trees and properties}
\subsection{Isomorphic rooted trees and properties}

\end{document}